\hypertarget{haskell}{%
\section{Haskell - Recursive Functions}\label{haskell}}

In Haskell, functions can also be defined in terms of themselves. Such
functions are called recursive.

\begin{itemize}
\tightlist
\item
  Some functions, such as factorial, are simpler to define in terms of
  other functions.
\item
  As we shall see, however, many functions can naturally be defined in
  terms of themselves.
\item
  Properties of functions defined using recursion can be proved using
  the simple but powerful mathematical technique of induction.
\end{itemize}

\hypertarget{recursion-on-lists}{%
\subsection{Recursion on Lists}\label{recursion-on-lists}}

\begin{lstlisting}[language=Haskell]
product :: Num a -> [a] -> a
product [] = 1
product (n:ns) = n * product ns

In execution:

product [2,3,4]
2 * product [3,4]
2 * (3 * product [4])
2 * (3 * (4 * product []))
2 * (3 * (4 * 1))
24

--or
-- The maximum function takes a list of things that can be ordered (e.g. instances of the Ord typeclass) and returns the biggest of them.

maximum' :: (Ord a) => [a] -> a  
maximum' [] = error "maximum of empty list"  
maximum' [x] = x  
maximum' (x:xs)   
    | x > maxTail = x  
    | otherwise = maxTail  
    where maxTail = maximum' xs
\end{lstlisting}

\hypertarget{multiple-arguments}{%
\subsection{Multiple Arguments}\label{multiple-arguments}}

Functions with more than one argument can also be defined using
recursion.

\begin{lstlisting}[language=Haskell]
zip :: [a] -> [b] -> [(a,b)]
zip []    _       = []
zip  _    []      = []
zip (x:xs) (y:ys) = (x,y) : zip xs ys
\end{lstlisting}

\subsection{Some Examples}

\begin{lstlisting}[language=Haskell]
take' :: (Num i, Ord i) => i -> [a] -> [a]  
take' n _  
    | n <= 0   = []  
take' _ []     = []  
take' n (x:xs) = x : take' (n-1) xs 

reverse' :: [a] -> [a]  
reverse' [] = []  
reverse' (x:xs) = reverse' xs ++ [x]
\end{lstlisting}

\clearpage
\subsection{Quick Sort}

We have a list of items that can be sorted. Their type is an instance of the \textbf{Ord} typeclass. The edge condition is the empty list. Now a sorted list is a list that has all the values smaller than (or equal to) the head of the list in front (and those values are sorted), then comes the head of the list in the middle and then come all the values that are bigger than the head (they're also sorted).

\begin{lstlisting}[language=Haskell]
quicksort :: (Ord a) => [a] -> [a]  
quicksort [] = []  
quicksort (x:xs) =   
    let smallerSorted = quicksort [a | a <- xs, a <= x]  
        biggerSorted = quicksort [a | a <- xs, a > x]  
    in  smallerSorted ++ [x] ++ biggerSorted  
\end{lstlisting}

So if we have, say [5,1,9,4,6,7,3] and we want to sort it, this algorithm will first take the head, which is 5 and then put it in the middle of two lists that are smaller and bigger than it. So at one point, you'll have [1,4,3] ++ [5] ++ [9,6,7]. We know that once the list is sorted completely, the number 5 will stay in the fourth place since there are 3 numbers lower than it and 3 numbers higher than it. Now, if we sort [1,4,3] and [9,6,7], we have a sorted list! We sort the two lists using the same function. Eventually, we'll break it up so much that we reach empty lists and an empty list is already sorted in a way, by virtue of being empty.

\clearpage