\hypertarget{haskell}{%
\section{Haskell - Interpreter}\label{haskell}}

\begin{tcolorbox}[colback=red!5!white,colframe=red!75!black]
TODO: Ich verstehe nicht ganz, was der Interpeter mit Haskell zu tun hat bzw. für was wir das gemacht haben.
\end{tcolorbox}

\hypertarget{abstract-operator}{%
\subsection{Abstract Operator}\label{abstract-operator}}

Instead of using expressions directly, one can replace them with
abstract syntax. For instance instead of using arithmetic operators, one
can define a data type `ArithOperator'.

\begin{lstlisting}[language=Haskell]
type Ident = String

data ArithOperator
= Times
| Div
| Mod
| Plus
| Minus
deriving (Eq, Show)

data ArithExpr
= LitAExpr Int
| IdAExpr Ident
| DyaAExpr ArithOperator ArithExpr ArithExpr

evalAOpr :: ArithOperator -> (Value -> Value -> Value)
evalAOpr Times = (*)
evalAOpr Div = div
evalAOpr Mod = mod
evalAOpr Plus = (+)
evalAOpr Minus = (-)

testExp1 = DyaAExpr Plus (LitAExpr 1) (DyaAExpr Times (LitAExpr 2) (LitAExpr 3))
\end{lstlisting}

\hypertarget{concrete-versus-abstract-syntax}{%
\subsubsection{Concrete Versus Abstract
Syntax}\label{concrete-versus-abstract-syntax}}

\begin{itemize}
\tightlist
\item
  Concrete syntax is easier to read for humans
\item
  Abstract syntax is easier to process for machines
\end{itemize}

\hypertarget{semantic-values}{%
\subsection{Semantic values}\label{semantic-values}}

\begin{lstlisting}[language=Haskell]
type State = Ident -> Value
\end{lstlisting}

\begin{itemize}
\tightlist
\item
  a state is modelled as a function mapping identifiers to values
\item
  example: the function s0 mapping ``x'' to 5, ``y'' to 7, ``z'' to 9,
  and everything else to 42 (or to an error)
\end{itemize}

\clearpage
\hypertarget{updates}{%
\subsubsection{updateS}\label{updates}}

\begin{lstlisting}[language=Haskell]
readS :: State -> Ident -> Value
readS s ident = s ident

updateS :: State -> (Ident, Value) -> State
updateS s (ident, val) ident'
| ident' == ident = val
| otherwise = s ident'
\end{lstlisting}

updateS s0 (``y'', 17) yields the function s1 mapping ``x'' to 5, ``y''
to 17, ``z'' to 9, and everything else to 42 (or to an error)

\clearpage