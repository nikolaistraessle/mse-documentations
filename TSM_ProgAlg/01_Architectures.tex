\hypertarget{architectures-of-parallel-infrastructures}{%
\section{Architectures of Parallel
Infrastructures}\label{architectures-of-parallel-infrastructures}}

\hypertarget{implicit-vs.explicit-parallelism}{%
\subsection{Implicit vs.~explicit
parallelism}\label{implicit-vs.explicit-parallelism}}

\textbf{Implicit} means, processors have multiple functional units and
execute multiple instructions in the same cycle. e.g.

\begin{itemize}
\tightlist
\item
  pipelining
\item
  superscalar execution (more than one instruction fetcher)
\item
  very long instruction word processors
\end{itemize}

\textbf{Explicit} parallelism means the program must specify concurrency
(control structure) and interaction (communication model) between
concurrent subtasks.

\hypertarget{parallel-programming-models}{%
\subsection{Parallel programming
models}\label{parallel-programming-models}}

\begin{itemize}
\tightlist
\item
  Process based models
\item
  Lightweight processes and Threads
\item
  Parallel programming language with syntax to specify parallelism

  \begin{itemize}
  \tightlist
  \item
    Not really common or used anymore
  \end{itemize}
\item
  Directive based programming models

  \begin{itemize}
  \tightlist
  \item
    extend the threaded model by facilitating creation and
    synchronization of threads
  \item
    Examples: Open MP, Linda, POP-C++
  \end{itemize}
\end{itemize}

\hypertarget{parallel-machine-model}{%
\subsubsection{Parallel Machine Model}\label{parallel-machine-model}}

\begin{itemize}
\tightlist
\item
  PRAM (Parallel Random Access Machine)

  \begin{itemize}
  \tightlist
  \item
    consists of p processors and a global memory of unbounded size that
    is uniformly accessible to all processors
  \item
    processors share a common clock but may execute different
    instructions in each cycle
  \end{itemize}
\end{itemize}

There are following memory access handling variatens:

\begin{itemize}
\tightlist
\item
  Exclusive-read, exclusive-write (EREW)
\item
  Concurrent-read, exclusive-write (CREW)
\item
  Exclusive-read, concurrent-write (ERCW)
\item
  Concurrent-read, concurrent-write (CRCW)
\end{itemize}

But what does concurrent write mean?

\begin{itemize}
\tightlist
\item
  Common: write only if all values are identical
\item
  Arbitrary: write the data from a randomly selected processor
\item
  Priority: follow a predetermined priority order
\item
  Sum: write the sum of all data items.
\end{itemize}

\hypertarget{different-grains-of-parallelism}{%
\subsection{Different grains of
parallelism}\label{different-grains-of-parallelism}}

\begin{itemize}
\tightlist
\item
  \textbf{Granularity}: the \textbf{ratio} of computation to
  communication

  \begin{itemize}
  \tightlist
  \item
    periods of computation are separated from periods of communication
    by synchronization events
  \item
    the parallel programmer has to select the right granularity to
    benefit from the underlying platform
  \end{itemize}
\item
  Chunking

  \begin{itemize}
  \tightlist
  \item
    How do I split the work?
  \item
    determining the amount of \textbf{data to assign} to each task
    (chunk or grain size)
  \end{itemize}
\end{itemize}

\begin{tcolorbox}[colback=red!5!white,colframe=red!75!black]
The best performance depends on the algorithm and the used hardware environment \\
\textbf{General rule}: increase chunk size if the communication overhead is too large
\end{tcolorbox}

\hypertarget{trade-offs-with-the-selected-chunk-size}{%
\subsubsection{Trade-offs with the selected chunk
size}\label{trade-offs-with-the-selected-chunk-size}}

\begin{itemize}
\tightlist
\item
  Fine-grained Parallelism

  \begin{itemize}
  \tightlist
  \item
    \textbf{better load-balancing}
  \item
    low arithmetic intensity
  \item
    may not have enough work to hide long-duration asynchronous
    communication
  \item
    too fine granularity leads to too \textbf{much overhead} (for
    communication)
  \end{itemize}
\item
  Coarse-grained Parallelism

  \begin{itemize}
  \tightlist
  \item
    high arithmetic intensity
  \item
    complete applications can serve as the grain of parallelism
  \item
    more difficult to load balance efficiently
  \end{itemize}
\end{itemize}

\textit{Irgendwie haben wir nicht mehr weitergemacht. Fehlende Kapitel sind: Control structure of parallel platforms, Communication model of parallel platforms,  Interconnection networks, Network topologies, Communication costs in parallel systems}

\clearpage