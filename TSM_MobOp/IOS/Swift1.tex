\section{Swift}

\begin{breakbox}
\begin{itemize}
\tightlist
\item
  Safety (e.g., Optionals, Definitive Initialization, etc)
\item
  Readability
\item
  Progressive Disclosure
\item
  Interoperability with Objective-C
\end{itemize}
\end{breakbox}

\begin{breakbox}
\boxtitle{Types}

\begin{itemize}
\tightlist
\item
  The following types are declared in the Swift Standard Library:

  \begin{itemize}
  \tightlist
  \item
    Int, Float, Double
  \item
    Bool
  \item
    String
  \item
    Array, Set, Dictionary\textless{}K, V\textgreater{}
  \end{itemize}
\item
  All of these types have value semantics (Wenn man z.B. eine Variable
  hat und man weist sie einer anderen Variable zu, dann ist das eine
  Kopie)
\item
  Some of these types use copy-on-write in order to be efficient
\item
  They are nominal types $\Rightarrow$ can be extended
\end{itemize}
\end{breakbox}

\begin{breakbox}
\boxtitle{let}

\textit{let hat definierte Standard-Typen, die verwendet werden wenn man
ein let initialisiert. Für Floating-Point Zahlen ist dies z.B. double.
Bei der Zuweisung lässt sich jedoch der Wert casten.}

\begin{lstlisting}[language=swift]
let c: Float = 4.5
print(type(of: c)) // changes the default type to float instead of double
\end{lstlisting}
\end{breakbox}

\begin{breakbox}
\boxtitle{Arrays}

\begin{lstlisting}[language=swift]
let ints1 = [1, 2, 3]
let strs = Array(repeating: "Hi", count: 5) // Array<String> / [String]
for s in strs { print(s) }
for (i, s) in strs.enumerated() {   // enumerated() returns a sequence
    print(i, s)                     // of (Int, String) tuples
}
print(arr.first)            //Optional(1)
print(arr.last)             //Optional(5)
print(arr.prefix(3))        //erste drei [1, 2, 3]
print(arr.dropFirst())      //[2, 3, 4, 5]
print(arr.randomElement())
print(arr.shuffled())
\end{lstlisting}
\end{breakbox}

\begin{breakbox}
\boxtitle{Dictionaries}

\begin{lstlisting}[language=swift]
let populationCounts = ["Switzerland": 8_081_000,
                        "Germany": 80_620_000,
                        "France": 66_030_000]

for (country, count) in populationCounts {
    print("\(country): \(count) people")
}

print(populationCounts["Switzerland"])  // Optional(8081000)

ageMap["John"] = nil    //removes an existing entry

for c in "test" {
    //Falls Eintrag nicht vorhanden, 0 hineinschreiben und direkt inkrementieren. Keine Ueberpruefung
    characterCounts[c, default: 0] += 1
}
\end{lstlisting}
\end{breakbox}

\begin{breakbox}
\boxtitle{Tuples}

\begin{itemize}
\tightlist
\item
  Group multiple values into a single compound value
\item
  There are no single-element tuples
\item
  The empty tuple () is a valid type
\end{itemize}

\begin{lstlisting}[language=swift]
let john = (33, "John")
print("(john.1) is (john.0).")

let dora1 = (age: 26, name: "Dora1")
print("(dora1.name) is (dora1.age).")
\end{lstlisting}
\end{breakbox}

\columnbreak
\begin{breakbox}
\boxtitle{Funktionstypen}

\begin{itemize}
\tightlist
\item
  Examples of function types:

  \begin{itemize}
  \tightlist
  \item
    (Int) -> Int $\Rightarrow$ takes an Int and returns an Int
  \item
    (String, Double) -> () $\Rightarrow$ takes a String and a Double
    and returns nothing
  \item
    () -> (Int, Int) $\Rightarrow$ takes no arguments and returns a
    pair of Ints
  \end{itemize}
\item
  Parentheses around argument types are always required. Return type
  should only have parentheses if you are returning a tuple
\end{itemize}

\begin{lstlisting}[language=swift]
func f1() {}
func f2(_ x: Int) -> Int { return x }
func f3(_ x: Int, _ y: Int) {}
func f4() -> (Int, Int) { return (0, 0) }
\end{lstlisting}
\end{breakbox}

\subsection{Statements}

\begin{breakbox}
\boxtitle{If Statement}

Die runden Klammern sind nicht unbedingt nötig, aber die geschweiften
schon! Es ist möglich, mehrere Statements mit einem Komma zu verknüpfen
(automatisch AND).
\end{breakbox}

\begin{breakbox}
\boxtitle{Switch Statement}


\begin{lstlisting}[language=swift]
//Kein Break
//Unterstuetzt pattern matching

let peopleCount = 42
switch peopleCount {
    case 0:
        print("no people")
    default:
        print("lots of people")
}
\end{lstlisting}
\end{breakbox}

\begin{breakbox}
\boxtitle{Loop Statements}

\begin{lstlisting}
for n in numbers {
    print(n)
}

//prints only the even numbers
for n in numbers where n % 2 == 0 {
    print(n)
}
\end{lstlisting}
\end{breakbox}

\begin{breakbox}
\boxtitle{Guard, Defer \& Error Handling}

\begin{lstlisting}[language=swift]
enum FileError: Error {
    case notFound
    case unknownEncoding
}

func readFile(at path: String) throws -> String {
    guard let file = FileHandle(forReadingAtPath: path) else { throw FileError.notFound }
    defer { file.closeFile() }
    let data = file.readDataToEndOfFile()
    guard let content = String(data: data, encoding: .utf8) else { throw FileError.unknownEncoding }
    return content
}

do {
    let content = try readFile(at: "/path/to/file.txt")
    print(content)
}
catch FileError.notFound { print("Error: File not found") }
catch FileError.unknownEncoding { print("Error: Unknown encoding") }
\end{lstlisting}

\textbf{Optional try:}
\begin{lstlisting}[language=swift]
//Liefert ein Optional und ist nil, wenn ein Fehler geworfen wird.
let content = try? readFile(at: "/path/to/file.txt")
\end{lstlisting}
\end{breakbox}

\subsection{Functions}

\begin{breakbox}
\begin{itemize}
\tightlist
\item
  Free functions are supported (e.g., min() and max() are free
  functions)
\item
  Parameters have an internal and an external name
\end{itemize}

\begin{lstlisting}[language=swift]
//person ist intern und extern derselbe Name
//from ist der externe Name, intern heisst dies hometown
func greet(person: String, from hometown: String) {
    print("Hello, \(person) from \(hometown)!")
}

//Mit '_' wird der externe Name komplett unterdrueckt.
func square(_ n: Int) -> Int {
    return n * n
}
\end{lstlisting}
\end{breakbox}

\begin{breakbox}
\boxtitle{Higher-Order Functions}

Higher-Order Functions are functions, that take a function as an argument or return a function.
\begin{lstlisting}[language=swift]
func makeMultiplier(factor: Int) -> (Int) -> Int {
    func multiplier(n: Int) -> Int {
        return factor * n
    }
    return multiplier
}

let multiplyByTwo = makeMultiplier(factor: 2)
let multiplyByFive = makeMultiplier(factor: 5)
let numbers = [1, 2, 3, 4, 5]
print(numbers.map(multiplyByTwo))   // [2, 4, 6, 8, 10]
print(numbers.map(multiplyByFive))  // [5, 10, 15, 20, 25]
\end{lstlisting}
\end{breakbox}

\begin{breakbox}
\boxtitle{Generic Functions}

\begin{lstlisting}[language=swift]
//Wir sagen, dass die Parameter Comparable sein muessen.
func _min<T : Comparable>(_ x: T, _ y: T) -> T {
    return y < x ? y : x
}
print(_min(1, 2))// 1
print(_min("hello", "hallo")) //"hallo"
\end{lstlisting}
\end{breakbox}

\begin{breakbox}
\boxtitle{Inout Parameters}

1. When the function is called,
the value of the argument is copied. 2. In the body of the function, the
copy is modified. 3. When the function returns, the copy's value is
assigned to the original argument.

\end{breakbox}

\begin{breakbox}
\boxtitle{Closures}

\begin{itemize}
\tightlist
\item
  Closures are expressions
\item
  Anonymous functions
\end{itemize}

\begin{lstlisting}[language=swift]
let numbers = [1, 2, 3, 4, 5]
let squaredNumbers = numbers.map({ (n: Int) -> Int in
    return n * n
})
print(squaredNumbers)       // [1, 4, 9, 16, 25]
\end{lstlisting}
\end{breakbox}

\begin{breakbox}
\boxtitle{Closure Capturing}

\begin{lstlisting}[language=swift]
func makeCounter() -> () -> Int {
    var count = 0
    return {
    // local variable 'count' is captured by closure
        count += 1
        return count
    }
}

let c1 = makeCounter()
let c2 = makeCounter()
print(c1())  //1
print(c1())  //2
print(c2())  //1
print(c1())  //3
\end{lstlisting}
\end{breakbox}

