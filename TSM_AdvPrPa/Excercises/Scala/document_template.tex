\documentclass[a4paper, 10pt, fleqn]{article}
%twoside
\usepackage[utf8]{inputenc}
\usepackage[T1]{fontenc}
\usepackage{textcomp}
\usepackage{lmodern}
\usepackage[ngerman]{babel}
\usepackage[nottoc]{tocbibind}
\usepackage{enumerate}
\usepackage{xcolor}
\usepackage{pdfpages}
\usepackage{amsmath}
\usepackage{graphicx}
\usepackage{geometry}
\usepackage{lastpage}
\usepackage[hyphens]{url}
\usepackage{hyperref}
\usepackage{listings}
\usepackage{float}
\usepackage{amssymb}
\usepackage{placeins}
\usepackage{color,soul}
\usepackage{verbatim}
\usepackage{pdflscape}
\usepackage{wrapfig}
\usepackage{paralist}
\usepackage{multicol}
\usepackage{tabu}
\usepackage{tocloft}
\usepackage{soul}
\usepackage[most]{tcolorbox}
\usepackage{ulem}
\usepackage{fancyhdr}
\usepackage{inconsolata}

%Tabelle
\usepackage{booktabs}% nicer horizontal lines$
\usepackage{longtable}
\usepackage{tabularx,colortbl}

%Twoside spacing fix
%\raggedbottom

\renewcommand*{\lstlistoflistings}{%
  \begingroup
  \tocchapter
  \tocfile{\lstlistlistingname}{lol}
  \endgroup
}

\renewcommand*{\listoffigures}{%
  \begingroup
  \tocchapter
  \tocfile{\listfigurename}{lof}
  \endgroup
}

\restylefloat{figure}

\newcommand{\code}[1]{\texttt{#1}}

\renewcommand*{\listoffigures}{%
  \begingroup
  \tocchapter
  \thispagestyle{fancy}
  \tocfile{\listfigurename}{lof}
  \endgroup
}

\geometry{left=3cm, top=3cm, bottom=3cm, right=2cm}

\hypersetup{
    colorlinks,
    linkcolor=black,
    citecolor=black,
    urlcolor=black
}
\tocloftpagestyle{fancy}

\pagestyle{fancy}
\fancyhf{}
\rhead{Diego Bienz}
\lhead{TSM - AdvPrPa - Scala}
\rfoot{Seite \thepage\ von \pageref{LastPage}}

\definecolor{bluekeywords}{rgb}{0.13,0.13,1}
\definecolor{greencomments}{rgb}{0,0.5,0}
\definecolor{redstrings}{rgb}{0.9,0,0}

\lstset{language=Haskell,
  mathescape=false,
  frame=single,
  numbers=left,                    
  numbersep=5pt,
  numberstyle=\tiny\color{gray},
  rulecolor=\color{black},
  showspaces=false,
  showtabs=false,
  breaklines=true,
  showstringspaces=false,
  breakatwhitespace=true,
  escapeinside={(*@}{@*)},
  tabsize=1,
  commentstyle=\color{greencomments},
  keywordstyle=\color{bluekeywords},
  stringstyle=\color{redstrings},
  basicstyle=\linespread{1}\ttfamily\footnotesize
}

\lstset{language=scala,
  mathescape=false,
  frame=single,
  numbers=left,                    
  numbersep=5pt,
  numberstyle=\tiny\color{gray},
  rulecolor=\color{black},
  showspaces=false,
  showtabs=false,
  breaklines=true,
  showstringspaces=false,
  breakatwhitespace=true,
  escapeinside={(*@}{@*)},
  tabsize=1,
  commentstyle=\color{greencomments},
  keywordstyle=\color{bluekeywords},
  stringstyle=\color{redstrings},
  basicstyle=\linespread{1}\ttfamily\footnotesize
}

\lstdefinelanguage{swift}
{
  morekeywords={
    func,if,then,else,for,in,while,do,switch,case,default,where,break,continue,fallthrough,return,
    typealias,struct,class,enum,protocol,var,func,let,get,set,willSet,didSet,inout,init,deinit,extension,
    subscript,prefix,operator,infix,postfix,precedence,associativity,left,right,none,convenience,dynamic,
    final,lazy,mutating,nonmutating,optional,override,required,static,unowned,safe,weak,internal,
    private,public,is,as,self,unsafe,dynamicType,true,false,nil,Type,Protocol,
  },
  morecomment=[l]{//}, % l is for line comment
  morecomment=[s]{/*}{*/}, % s is for start and end delimiter
  morestring=[b]" % defines that strings are enclosed in double quotes
}

% Einrücken zu Beginn von neuem Absatz unterdrücken
\setlength{\parindent}{0pt}

% Zeilenabstand einstellen
\usepackage{setspace}
\makeatletter
\newcommand{\MSonehalfspacing}{%
  \setstretch{1.44}%  default
  \ifcase \@ptsize \relax % 10pt
    \setstretch {1.448}%
  \or % 11pt
    \setstretch {1.399}%
  \or % 12pt
    \setstretch {1.433}%
  \fi
}
\newcommand{\MSdoublespacing}{%
  \setstretch {1.92}%  default
  \ifcase \@ptsize \relax % 10pt
    \setstretch {1.936}%
  \or % 11ptEB
    \setstretch {1.866}%
  \or % 12pt
    \setstretch {1.902}%
  \fi
}
\makeatother
\MSonehalfspacing

\newcommand\tightlist
