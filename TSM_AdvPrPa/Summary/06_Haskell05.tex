\hypertarget{haskell}{%
\section{Haskell - List Comprehensions}\label{haskell}}

In Haskell, a comprehension notation can be used to construct new lists
from old lists.

\begin{lstlisting}[language=Haskell]
ghci> [x*2 | x <- [1..10], x*2 >= 12]  
[12,14,16,18,20]

boomBangs xs = [ if x < 10 then "BOOM!" else "BANG!" | x <- xs, odd x]
ghci> boomBangs [5..15]
["BOOM!","BOOM!","BOOM!","BANG!","BANG!","BANG!"]

[(x,y) | y <- [4,5], x <- [1,2,3]]
[(1,4),(2,4),(3,4),(1,5),(2,5),(3,5)]

ghci> let nouns = ["hobo","frog","pope"]  
ghci> let adjectives = ["lazy","grouchy","scheming"]  
ghci> [adjective ++ " " ++ noun | adjective <- adjectives, noun <- nouns]  
["lazy hobo","lazy frog","lazy pope","grouchy hobo","grouchy frog","grouchy pope","scheming hobo","scheming frog","scheming pope"]   
\end{lstlisting}

\begin{itemize}
\tightlist
\item
  The expression x ← {[}1..5{]} is called a generator, as it states how
  to generate values for x
\item
  Comprehensions can have multiple generators, separated by commas
\item
  Changing the order of the generators changes the order of the elements
  in the final list
\item
  Multiple generators are like nested loops, with later generators as
  more deeply nested loops whose variables change value more frequently
\end{itemize}

\hypertarget{dependant-generators}{%
\subsubsection{Dependant Generators}\label{dependant-generators}}

Later generators can depend on the variables that are introduced by
earlier generators. Using a dependant generator we can define the
library function that concatenates a list of lists.

\begin{lstlisting}[language=Haskell]
concat :: [[a]] -> [a]
concat xss = [x | xs <- xss, x <- xs]

concat [[1,2,3],[4,5],[6]]
[1,2,3,4,5,6]
\end{lstlisting}

\hypertarget{guards}{%
\subsubsection{Guards (Predicates)}\label{guards}}

List comprehensions can use guards to restrict the values produced by
earlier generators.

\begin{lstlisting}[language=Haskell]
ghci> [x | x <- [1..10], even x]
[2,4,6,8,10]

--Multiple predicates are seperated by comma
ghci> [ x | x <- [10..20], x /= 13, x /= 15, x /= 19]  
[10,11,12,14,16,17,18,20] 
\end{lstlisting}

\hypertarget{the-zip-function}{%
\subsection{The Zip Function}\label{the-zip-function}}

A useful library function is zip, which maps two lists to a list of
pairs of their corresponding elements.

\begin{lstlisting}[language=Haskell]
zip :: [a] -> [b] -> [(a,b)]
> zip ['a','b','c'] [1,2,3,4]
[('a',1),('b',2),('c',3)]
\end{lstlisting}

\clearpage
\hypertarget{string-comprehensions}{%
\subsection{String Comprehensions}\label{string-comprehensions}}

\begin{itemize}
\tightlist
\item
  A string is a sequence of characters enclosed in double quotes.
  Internally, however, strings are represented as lists of characters.
\item
  Because strings are just special kinds of lists, any polymorphic
  function that operates on lists can also be applied to strings.
\item
  Similarly, list comprehensions can also be used to define functions on
  strings, such counting how many times a character occurs in a string:
\end{itemize}

\begin{lstlisting}[language=Haskell]
count :: Char -> String -> Int
count x xs = length [x' | x' <- xs, x == x']

--Own version of length
length' xs = sum [1 | _ <- xs]   

removeNonUppercase st = [ c | c <- st, c `elem` ['A'..'Z']] 
ghci> removeNonUppercase "Hahaha! Ahahaha!"  
"HA"
\end{lstlisting}

\clearpage