\hypertarget{haskell}{%
\section{Haskell - Defining Functions}\label{haskell}}

\hypertarget{conditional-expressions}{%
\subsection{Conditional Expressions}\label{conditional-expressions}}

As in most programming languages, functions can be defined using
conditional expressions.

\begin{lstlisting}[language=Haskell]
abs :: Int -> Int
abs n = if n >= 0 then n else -n
\end{lstlisting}

abs takes an integer n and returns n if it is non-negative and -n
otherwise.

Conditional expressions can also be nested:

\begin{lstlisting}[language=Haskell]
signum :: Int -> Int
signum n = if n < 0 then -1 else
if n == 0 then 0 else 1
\end{lstlisting}

\begin{tcolorbox}[colback=red!5!white,colframe=red!75!black]
In Haskell, conditional expressions must always have an else branch.
\end{tcolorbox}

\hypertarget{guarded-equations}{%
\subsubsection{Guarded Equations}\label{guarded-equations}}

As an alternative to conditionals, functions can also be defined using
guarded equations

\begin{lstlisting}[language=Haskell]
abs n | n >= 0 = n
      | otherwise = -n
\end{lstlisting}

Guarded equations can be used to make definitions involving multiple
conditions easier to read.

\hypertarget{pattern-matching}{%
\subsection{Pattern Matching}\label{pattern-matching}}

Many functions have a particularly clear definition using pattern matching on their arguments.

\begin{lstlisting}[language=Haskell]
lucky :: (Integral a) => a -> String  
lucky 7 = "LUCKY NUMBER SEVEN!"  
lucky x = "Sorry, you're out of luck, pal!" 
\end{lstlisting}

When you call lucky, the patterns will be checked from top to bottom and when it conforms to a pattern, the corresponding function body will be used. The only way a number can conform to the first pattern here is if it is 7. If it's not, it falls through to the second pattern, which matches anything and binds it to x.

\begin{lstlisting}[language=Haskell]
factorial :: (Integral a) => a -> a  
factorial 0 = 1  
factorial n = n * factorial (n - 1) 

addVectors :: (Num a) => (a, a) -> (a, a) -> (a, a)  
addVectors a b = (fst a + fst b, snd a + snd b)  
--addVectors (x1, y1) (x2, y2) = (x1 + x2, y1 + y2)
\end{lstlisting}

\begin{tcolorbox}[colback=red!5!white,colframe=red!75!black]
The underscore symbol \_ is a wildcard pattern that matches any argument value.
Patterns are matched in order.
Patterns may not repeat variables
\end{tcolorbox}

\hypertarget{list-patterns}{%
\subsubsection{List Patterns}\label{list-patterns}}

Since [1,2,3] is just syntactic sugar for 1:2:3:[], you can also use the former pattern. A pattern like x:xs will bind the head of the list to x and the rest of it to xs, even if there's only one element so xs ends up being an empty list. 

\begin{tcolorbox}[colback=red!5!white,colframe=red!75!black]
[1,2,3,4] means internal actually 1:(2:(3:(4:[]))). <- Syntactic Sugar
\end{tcolorbox}

Functions on lists can be defined using x:xs patterns. For more operations on lists refer to chapter \ref{sec:Operationonlists}.

\begin{lstlisting}[language=Haskell]
head :: [a] -> a
head (x:_) = x
tail :: [a] -> [a]
tail (_:xs) = xs

head' :: [a] -> a  
head' [] = error "Can't call head on an empty list, dummy!"  
head' (x:_) = x  

ghci> let xs = [(1,3), (4,3), (2,4), (5,3), (5,6), (3,1)]  
ghci> [a+b | (a,b) <- xs]  
[4,7,6,8,11,4]  
\end{lstlisting}

\subsection{Guard}

Whereas patterns are a way of making sure a value conforms to some form and deconstructing it, guards are a way of testing whether some property of a value (or several of them) are true or false. Guards are indicated by pipes that follow a function's name and its parameters. A guard is basically a boolean expression. If it evaluates to True, then the corresponding function body is used. If it evaluates to False, checking drops through to the next guard and so on. Many times, the last guard is otherwise. otherwise is defined simply as otherwise = True and catches everything. 

Guards can also be written inline, although I'd advise against that because it's less readable, even for very short functions.

\begin{lstlisting}[language=Haskell]
bmiTell :: (RealFloat a) => a -> a -> String  
bmiTell weight height  
    | bmi <= skinny = "You're underweight, you emo, you!"  
    | bmi <= 25.0 = "You're supposedly normal. Pffft, I bet you're ugly!"  
    | bmi <= 30.0 = "You're fat! Lose some weight, fatty!"  
    | otherwise   = "You're a whale, congratulations!"  
    where bmi = weight / height ^ 2 
          skinny = 18.5  

max' :: (Ord a) => a -> a -> a  
max' a b | a > b = a | otherwise = b  
\end{lstlisting}

We put the keyword \textbf{where} after the guards (usually it's best to indent it as much as the pipes are indented) and then we define several names or functions. These names are visible across the guards and give us the advantage of not having to repeat ourselves.

\clearpage
\subsection{Let Statement}

The form is let <bindings> in <expression>. The names that you define in the let part are accessible to the expression after the \textbf{in} part. Let bindings are expressions and are fairly local in their scope, they can't be used across guards (which \textit{where} can.

\begin{lstlisting}[language=Haskell]
cylinder :: (RealFloat a) => a -> a -> a  
cylinder r h = 
    let sideArea = 2 * pi * r * h  
        topArea = pi * r ^2  
    in  sideArea + 2 * topArea  
    
--Different variables are seperated with semikolon.
ghci> (let a = 100; b = 200; c = 300 in a*b*c, let foo="Hey "; bar = "there!" in foo ++ bar)  
(6000000,"Hey there!") 
\end{lstlisting}

\hypertarget{lambda-expressions}{%
\subsection{Lambda Expressions}\label{lambda-expressions}}

Functions can be constructed without naming the functions by using
lambda expressions.

\begin{lstlisting}[language=Haskell]
$\lambda$x -> x + x
\end{lstlisting}

The symbol $\lambda$ is the Greek letter lambda, and is typed at the keyboard as a backslash \textbackslash{}.

\begin{itemize}
\tightlist
\item
  Lambda expressions can be used to give a formal meaning to functions
  defined using currying
\item
  Lambda expressions are also useful when defining functions that return
  functions as results
\item
  Lambda expressions can be used to avoid naming functions that are only
  referenced once
\end{itemize}

\hypertarget{operator-sections}{%
\subsection{Operator Sections}\label{operator-sections}}

An operator written between its two arguments can be converted into a
curried function written before its two arguments by using parentheses.

\begin{lstlisting}[language=Haskell]
> 1+2
3
> (+) 1 2
3

or
> (1+) 2
3
\end{lstlisting}

In general, if $\oplus$ is an operator then functions of the form ($\oplus$), (x$\oplus$) and (y$\oplus$) are called sections.

\clearpage
