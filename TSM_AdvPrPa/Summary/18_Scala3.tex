\hypertarget{scala-types}{%
\section{Scala - More Types}\label{scala-types}}

\hypertarget{inner-types}{%
\subsection{Inner Types}\label{inner-types}}

Classes, traits and singleton objects can have inner types, i.e.

\begin{itemize}
\tightlist
\item
  Inner classes
\item
  Inner traits
\item
  Inner type members
\end{itemize}

\begin{lstlisting}[language=scala,mathescape=false]
class Outer {
    class InnerClass
    trait InnerTrait
    type InnerType = String
}
\end{lstlisting}

\hypertarget{path-dependent-types}{%
\subsubsection{Path-dependent Types}\label{path-dependent-types}}

To access an inner type you need an outer instance

\begin{lstlisting}[language=scala,mathescape=false]
class Outer {
    class Inner
    def put(inner: /* optional: this.*/Inner) {}
}
defined class Outer

scala> val outer1 = new Outer
outer1: Outer = Outer@6ef8fff2

scala> val inner1 = new outer1.Inner
inner1: outer1.Inner = Outer$Inner@11c7d2e3
\end{lstlisting}

The type of an inner instance depends on the outer instance

\begin{lstlisting}[language=scala,mathescape=false]
scala> val outer2 = new Outer
outer2: Outer = Outer@73a1deca

scala> val inner2 = new outer2.Inner
inner2: outer2.Inner = Outer$Inner@1a1dd898

scala> outer1.put(inner2)
<console>:14: error: type mismatch;
found   : outer2.Inner
required: outer1.Inner
\end{lstlisting}

The inner types share a common supertype, denoted by a type selection
Outer\#Inner

\begin{lstlisting}[language=scala,mathescape=false]
class Outer {
    class Inner
    def put(inner: Outer#Inner) {}
}
defined class Outer

scala> val inner1: Outer#Inner = new outer1.Inner
inner1: Outer#Inner = Outer$Inner@502c500d

scala> val inner2: Outer#Inner = new outer2.Inner
inner2: Outer#Inner = Outer$Inner@7e912427

scala> outer2.put(inner1)
\end{lstlisting}

\hypertarget{dependent-types}{%
\subsubsection{Dependent Types}\label{dependent-types}}

Dependent types are types which depend on a value. They can be emulated
with path dependent types.

\begin{lstlisting}[language=scala,mathescape=false]
object Arrays {
    case class OfLength(n: Int) {
        class Arr {
            private val a: Array[Int] = new Array[Int](n)
            def apply(x: Int) = a(x)
            def update(x: Int, y: Int): Unit = {a(x) = y}
        }
    }
    // The following types are two different types
    val ofLength4 = OfLength(4)
    val ofLength5 = OfLength(5)

    type ArrayOfLength4 = ofLength4.Arr
    type ArrayOfLength5 = ofLength5.Arr

    def f(a: ArrayOfLength5): Unit = { a(4) = 4 }

    val a4 = new ArrayOfLength4
    val a5 = new ArrayOfLength5

    f(a5) // <<= OK
    f(a4) // <<= type error
    //<console> :14: error: type mismatch;
    //found     : ofLength4.Arr
    //required  : ArrayOfLength5
    //(which expands to) ofLength5.Arr
}
\end{lstlisting}

\hypertarget{singleton}{%
\subsection{Singleton}\label{singleton}}

Singleton type is a type which represents a reference. Given any
reference v, the type v.type has two values: v and null.

\begin{lstlisting}[language=scala,mathescape=false]
scala> class X
defined class X
scala> val x = new X
x: X = X@12d05dde
scala> var y: x.type = x
y: x.type = X@12d05dde
scala> y = null
y: x.type = null
scala> y = new X
<console>   :13: error: type mismatch;
found       : X
required    : x.type
\end{lstlisting}

\clearpage
\hypertarget{application-1-method-chaining-fluent-interfaces}{%
\subsubsection{Application 1: Method Chaining (fluent
interfaces)}\label{application-1-method-chaining-fluent-interfaces}}

Fluent interface is a method for designing object oriented APIs based
extensively on method chaining with the goal of making the readability
of the source code close to that of ordinary written prose.

\begin{lstlisting}[language=scala,mathescape=false]
class Document {
    def setTitle(title: String) = { ...; this }
    def setAuthor(author: String) = { ...; this }
    ...
}

article.setAuthor("Horstmann").setTitle("Scala")

class Book extends Document {
    def addChapter(chapter: String) = { ...; this }
    ...
}

val book = new Book()
book.setTitle("Scala").addChapter(chapter1)
error: value addChapter is not a member of Document
\end{lstlisting}

With an additional statement, one can preserve the original type in the
class:

\begin{lstlisting}[language=scala,mathescape=false]
class Document {
    def setTitle(title: String): this.type = { ...; this }
    def setAuthor(author: String): this.type = { ...; this }
    ...
}
class Book extends Document {
    def addChapter(chapter: String) = { ...; this }
    ...
}

scala> book.setTitle("Scala").addChapter("ch1")
res32: Book = Book@5213fe32
scala> book.setTitle("Scala")
res33: book.type = Book@5213fe3
\end{lstlisting}

\hypertarget{application-2-fluent-interfaces}{%
\subsubsection{Application 2: Fluent
interfaces}\label{application-2-fluent-interfaces}}

\begin{lstlisting}[language=scala,mathescape=false]
object Title
class Document {
    var title: String = ""
    private var useNextArgAs: Any = null
    def set(obj: Title.type) = { useNextArgAs = obj; this }
    def to(arg: String) = {
        if(useNextArgAs == Title) { title = arg; } // else ...
    }
    //...
}
val book = new Document
book.set(Title).to("Scala")

book set Title to "Scala"
//In scala it is possible to leave away the paranthesis and the dots, if there is only one parameter.
\end{lstlisting}

\clearpage
\hypertarget{implicits}{%
\subsection{Implicits}\label{implicits}}

\hypertarget{implicit-conversions}{%
\subsubsection{Implicit Conversions}\label{implicit-conversions}}

Implicit conversion function

\begin{itemize}
\tightlist
\item
  Method or function with one argument (method name is irrelevant)
\item
  Marked with the implicit keyword
\item
  The compiler knows how to compile the given type to the target type

  \begin{itemize}
  \tightlist
  \item
    e.g.~when actually a Fraction is required but the given type is
    integer, the compiler realizes that there is a converter.
  \end{itemize}
\end{itemize}

\begin{lstlisting}[language=scala,mathescape=false]
import scala.language.implicitConversions
implicit def int2Fraction(n: Int) = Fraction(n,1)

scala> val f1: Fraction = 2
f1: Fraction = Fraction(2,1)
scala> 3 * Fraction(4,5)
res2: Fraction = Fraction(12,5)
\end{lstlisting}

Another example

\begin{lstlisting}[language=scala,mathescape=false]
//Could also write "implicit class BlingString(string: String) {.." and leave the "implicit def ..." away.

class BlingString(string: String) {
    def bling = "*" + string + "*"
}
implicit def blingToString(s: String) = new BlingString(s)

scala> "Hello".bling
res1: String = *Hello*

scala> "Hello".bling.bling
res2: String = **Hello**
\end{lstlisting}

Instead of creating a new instance for every application, a value class
can be used

\begin{itemize}
\tightlist
\item
  Value classes avoid creation of objects
\item
  Value class has a single public val parameter that is the underlying
  runtime representation
\item
  Has only defs as members, but no vals, vars, or nested traits, classes
  or objects
\end{itemize}

\begin{lstlisting}[language=scala,mathescape=false]
implicit class BlingString(val string: String) extends AnyVal {
    def bling = "*" + string + "*"
}
\end{lstlisting}

\hypertarget{implicit-conversion-rules}{%
\subsubsection{Implicit Conversion
Rules}\label{implicit-conversion-rules}}

Implicit conversions are considered in three cases.

If the type of an expression differs from the expected type

\begin{lstlisting}[language=scala,mathescape=false]
val f: Fraction = 12
\end{lstlisting}

If a non-existent member is accessed

\begin{lstlisting}[language=scala,mathescape=false]
"hello".bling
\end{lstlisting}

If an object invokes a method whose parameters don't match the given
arguments

\begin{lstlisting}[language=scala,mathescape=false]
3 * Fraction(4,5) // Int.* does not accept a Fraction arg
\end{lstlisting}

In order to be applied, an implicit conversion must be in scope

Current scope: Identifiers accessible without prefix

\begin{itemize}
\tightlist
\item
  Local identifiers

  \begin{itemize}
  \tightlist
  \item
    Members of an enclosing scope
  \item
    Imported identifiers
  \end{itemize}
\item
  Implicit scope: Members of companion objects of associated types

  \begin{itemize}
  \tightlist
  \item
    The types in question (source or target type)
  \item
    All parts of a parameterized type, e.g.~A{[}B,C{]}
  \item
    All parts of a compound type, e.g.~A with B with C
  \end{itemize}
\item
  Precedence rules

  \begin{itemize}
  \tightlist
  \item
    Local implicits
  \item
    Imported implicits
  \item
    Companion object of the types
  \item
    Companion object of the type arguments of the type
  \end{itemize}
\end{itemize}

\hypertarget{implicit-parameters}{%
\subsubsection{Implicit Parameters}\label{implicit-parameters}}

\begin{itemize}
\tightlist
\item
  Use the keyword implicit to define an implicit parameter list
\item
  This only works for the last parameter list
\end{itemize}

\begin{lstlisting}[language=scala,mathescape=false]
case class Delimiters(left: String, right: String)

def quote(text: String)(implicit delims: Delimiters) = delims.left + text + delims.right

scala> quote("Bonjour")(Delimiters("<<", ">>")) // Guillemets
res1: String = <<Bonjour>>

scala> quote("Hello")
<console>:14: error: could not find implicit value for parameter
delims: Delimiters
\end{lstlisting}

For this, one can define which implicit types should be used, if nothing
is given with the function.

\begin{lstlisting}[language=scala,mathescape=false]
import java.io.PrintStream
import java.lang._
implicit val out = System.out

def log(msg: String)(implicit out: PrintStream) = out.println(msg)

scala> log("Implicits are cool")
Implicits are cool

scala> log("here is an error")(System.err)
here is an error
\end{lstlisting}

\hypertarget{type-classes}{%
\subsection{Type classes}\label{type-classes}}

Type classes are among the most powerful features in Haskell. They allow
you to define generic interfaces that provide a common feature set over
a wide variety of types. Type classes are at the heart of some basic
language features such as equality testing and numeric operators.

\begin{itemize}
\tightlist
\item
  Type classes define a group (class) of types which satisfy some
  contract (defined in a trait)
\item
  Type classes add ad-hoc polymorphism
\item
  Example: sort method on a list
\end{itemize}

\begin{lstlisting}[language=scala,mathescape=false]
scala> List(3,2,1).sorted
res12: List[Int] = List(1, 2, 3)
\end{lstlisting}

What constraint is defined on the type parameter?

\begin{itemize}
\tightlist
\item
  class List{[}A \textless{}: Comparable{[}A{]}{]} \{ \ldots{} \}
\item
  def sorted\href{implicit\%20ord:\%20math.Ordering\%5BB\%5D}{B
  \textgreater{}: A}: List{[}A{]}
\end{itemize}

\hypertarget{polymorphism}{%
\subsubsection{Polymorphism}\label{polymorphism}}

\begin{itemize}
\tightlist
\item
  Ad-hoc polymorphism

  \begin{itemize}
  \tightlist
  \item
    Different implementations depending on the parameter types
  \item
    Method overloading is one example of ad-hoc polymorphism
  \end{itemize}
\item
  Parametric polymorphism

  \begin{itemize}
  \tightlist
  \item
    Code to be used transparently with any number of new types
  \item
    Generics is an example of parametric polymorphism
  \end{itemize}
\item
  Subtype polymorphism
\item
  Subtyping allows a function to be written to take an object of a
  certain type T, but also work correctly with subtypes of T
  (=\textgreater{} Liskov substitution principle)
\end{itemize}

\hypertarget{monoid}{%
\subsubsection{Monoid}\label{monoid}}

\begin{lstlisting}[language=scala,mathescape=false]
trait Monoid[A] {
    def op(x: A, y: A) : A
    def unit : A
}

//Two implementations
implicit object stringMonoid extends Monoid[String] {
    def op(x: String, y: String) = x + y
    def unit = ""
}
implicit object addMonoid extends Monoid[Int] {
    def op(x: Int, y: Int) = x + y
    def unit = 0
}

//Object of Monoid
object Monoid {
    def apply[A](implicit m: Monoid[A]) = m
}

scala> Monoid[String]
res1: Monoid[String] = stringMonoid$@64fc6470
scala> Monoid[Int]
res2: Monoid[Int] = addMonoid$@680937c9
\end{lstlisting}

\hypertarget{use-monoids-for-a-list-accumulator}{%
\subsubsection{Use Monoids for a List
accumulator}\label{use-monoids-for-a-list-accumulator}}

\begin{lstlisting}[language=scala,mathescape=false]
def acc[A](list: List[A])(implicit m: Monoid[A]) : A = list.foldLeft(m.unit)(m.op)

scala> acc(List(1,2,3))
res1: Int = 6

scala> acc(List("Hello", "World"))
res2: String = HelloWorld

scala> acc(List(List(1), List(1,2), List(1,2,3)))
<console>:12: error: could not find implicit value for parameter m:
Monoid[List[Int]]
\end{lstlisting}

\hypertarget{conditional-implicits}{%
\subsubsection{Conditional implicits}\label{conditional-implicits}}

Implicit methods can themselves have implicit parameters. This is super
fancy!

\begin{lstlisting}[language=scala,mathescape=false]
implicit def listMonoid[A](implicit m: Monoid[A]) =
    new Monoid[List[A]] {
        def op(xs: List[A], ys: List[A]): List[A] =
            if(xs.isEmpty) ys
            else if (ys.isEmpty) xs
            else m.op(xs.head, ys.head) :: op(xs.tail, ys.tail)
        def unit = List()
    }
}

scala> acc(List(List(1), List(1,2), List(1,2,3)))
res3: List[Int] = List(3, 4, 3)
//The lists are added element-wise
//[1+1+1, 0+2+2, 0+0+3] = [3, 4, 3]
\end{lstlisting}

\clearpage
\hypertarget{shorthand-notation-for-implicit-parameters}{%
\subsubsection{Shorthand notation for implicit
parameters}\label{shorthand-notation-for-implicit-parameters}}

\begin{lstlisting}[language=scala,mathescape=false]
def acc[A](list: List[A])(implicit m: Monoid[A]) : A = list.foldLeft(m.unit)(m.op)

def acc[A : Monoid](list : List[A]) : A = {
    val m = Monoid[A]
    list.foldLeft(m.unit)(m.op)
}
\end{lstlisting}

\hypertarget{conclusion-to-type-classes}{%
\subsubsection{Conclusion to type
classes}\label{conclusion-to-type-classes}}

Type classes allow us to

\begin{itemize}
\tightlist
\item
  introduce polymorphic functions, extensible (by composition) after the
  original code has been written (by providing new implementations)

  \begin{itemize}
  \tightlist
  \item
    Example: Monoid for String, Int and List{[}A{]}
  \end{itemize}
\item
  introduce generic functions in terms of the prototypes assumed to
  exist

  \begin{itemize}
  \tightlist
  \item
    Example: Method acc
  \end{itemize}
\item
  Implicit resolution

  \begin{itemize}
  \tightlist
  \item
    Implicits are looked for in the local scope (contrary to Haskell)

    \begin{itemize}
    \tightlist
    \item
      Several different implementations can be provided for a particular
      type
    \item
      With imports the implicits to be used are controllable
    \end{itemize}
  \end{itemize}
\end{itemize}

\clearpage
\hypertarget{limitations}{%
\subsubsection{Limitations}\label{limitations}}

The following code shows three classes. Two B classes are derivated from
A and override their own show method.

\begin{lstlisting}[language=scala,mathescape=false]
trait Show { def show: String }
class A extends Show { override def show = "A" }
class B1 extends A { override def show = "B1" }
class B2 extends A { override def show = "B2" }

scala> val list = List(new A, new B1, new B2)
list: List[A] = List(A@16d48c9, B1@2dc584da, B2@2951bb0)
scala> list.map( x => x.show )
res1: List[String] = List(A, B1, B2)
\end{lstlisting}

This can also be done with type classes:

\begin{lstlisting}[language=scala,mathescape=false]
trait Showable[T] { def show(x: T): String }
object Showable {
    implicit object a extends Showable[A]{ def show(x: A) = "TC:A" }
    implicit object b1 extends Showable[B1]{ def show(x: B1) = "TC:B1"}
    implicit object b2 extends Showable[B2] { def show(x: B2) = "TC:B2" }
}
def show[T : Showable](x: T) = {
    implicitly[Showable[T]].show(x)
}
scala> list.map( x => show(x))
res2: List[String] = List(TC:A, TC:A, TC:A)
\end{lstlisting}

As you see, it prints everytime A. This is because the type is evaluated
at compile time and the type of the List is A.

\clearpage