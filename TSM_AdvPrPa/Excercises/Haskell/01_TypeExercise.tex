\hypertarget{haskell-types}{%
\section{Haskell Types}\label{haskell-types}}

\hypertarget{types-of-lists-and-tuples}{%
\subsection{Types of Lists and Tuples}\label{types-of-lists-and-tuples}}

After declare x = `x':

\begin{lstlisting}[language=Haskell]
x ==> Char
'x' ==> Char
"x" ==> [Char]
['x'] ==> [Char]
[x, 'x'] ==> [Char]
[x, x, x, x] ==> [Char]
['x', "x"] ==> error
[x, True] ==> error
[x == 'x', True] ==> [Bool]
[["True"]] ==> [[[Char]]]
[[ True, False], True] ==> error
[[ True, False], []] ==> [[Bool]]
('x') ==> Char
(x , 'x') ==> (Char, Char)
(x ,x, x, x) ==> (Char, Char, Char, Char)
('x', "x") ==> (Char, [Char])
(x , True) ==> (Char, Bool)
(x == 'x' ,True) ==> (Bool, Bool)
(("True")) ==> [Char]
((True ,False) ,True) ==> ((Bool, Bool), Bool)
((True, False), ()) ==> ((Bool, Bool), ())
\end{lstlisting}

\hypertarget{types-of-lists}{%
\subsection{Types of Lists}\label{types-of-lists}}

\begin{lstlisting}[language=Haskell]
After declare a = [True]:

a ==> [Bool]
a ++ a ++ [True] ==> [Bool]
a ++ [] ==> [Bool]
head a ==> Bool
tail a (*) ==> error
head 'x' ==> error
head "x" ==> Char
tail "x" (*) ==> error
"dimdi" !! 2 ==> Char (Gibt den dritten Buchstaben aus (0 indexiert))
"dimdi" ++ "ding" ==> [Char]
\end{lstlisting}

\hypertarget{types-of-functions-and-lists}{%
\subsection{Types of Functions and
Lists}\label{types-of-functions-and-lists}}

\begin{lstlisting}[language=Haskell]
f1 :: Int -> Int
f1 x = x^2 + x + 1
f2 :: Int -> Int
f2 x = 2 * x + 1

f1 ==> Int -> Int
f1 5 ==> Int
f1 f2 ==> error
f1 (f2 5) ==> Int
[f1 5, f2 6, 5, 6] ==> [Int]
[f1, f2, f1] ==> [Int -> Int]
[f1 5, f2] ==> error
(f1 5, f2) ==> (Int, Int -> Int)
([f1, f2, f1] !! 1) 3 ==> Int
([f1, f2, f1] !! 5) 3 (*) ==> error
\end{lstlisting}

\hypertarget{types-of-functions-with-currying}{%
\subsection{Types of Functions with
Currying}\label{types-of-functions-with-currying}}

\begin{lstlisting}[language=Haskell]
g1 :: Int -> Int -> Int -> Int
g1 x y z = x^2 + y^2 + z^2
g2 :: Int -> Int -> Int
g2 x y = 2*x + 2*y

g1  ==> Int -> Int -> Int -> Int
g1 2 ==> Int -> Int -> Int
g1 2 3 ==> Int -> Int
g1 2 3 4 ==> Int
g1 2 3 4 5 ==> eroor
(g1, g2) ==> (Int -> Int -> Int -> Int, Int -> Int -> Int)
(g1 2, g2) ==> (Int -> Int -> Int, Int -> Int -> Int)
(g1 2 3, g2 4) ==> (Int -> Int, Int -> Int)
(g1 2 3 4, g2 4 5) ==> (Int, Int)
[g1, g2] ==> error
[g1 2, g2] ==> [Int -> Int -> Int]
[g1 2 3, g2 4] ==> [Int -> Int]
[g1 2 3 4, g2 4 5] ==> [Int]
\end{lstlisting}

\clearpage