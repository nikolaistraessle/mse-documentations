\section{Storing Persistent Application Data}

\begin{breakbox}
\begin{itemize}
\tightlist
\item
  Shared Preferences\\
  Store private primitive data in key-value pairs in an external file
\item
  Internal Storage\\
  Store private data on the device memory
\item
  External Storage\\
  Store public data on the shared external storage
\item
  SQLite Databases\\
  Store structured data in a private database
\item
  Network Connection\\
  Store data over the web on your own server or in a cloud
\end{itemize}
\end{breakbox}

\begin{breakbox}
\boxtitle{Shared Preferences}

The \textbf{SharedPreferences} class provides a general framework that
allows you to save and retrieve persistent \textbf{key-value pairs of
primitive data types}.

\textbf{Get Methods:}
\begin{itemize}
    \item getPreferences() for one preference file
    \item getSharedPreferences() for
multiple preference files, distinguished by name.
\end{itemize}

\textbf{Save Methods:} \\
Call edit() to get the Editor, add values with methods such as
putBoolean(), putString(), \ldots{} and commit the new values with
\textbf{commit()}.

\clearpage
\begin{lstlisting}
public class Calc extends Activity {
    public static final String PREFS_NAME = "MyPrefsFile";

    @Override
    protected void onCreate(Bundle state){
        super.onCreate(state);
        . . .
        // Restore preferences
        SharedPreferences settings = getSharedPreferences(PREFS_NAME, 0); // 0: Permission
        boolean silent = settings.getBoolean("silentMode", false); // false: Default value
        setSilent(silent);
    }

    @Override
    protected void onStop(){
        super.onStop();
        // We need an Editor object to make preference changes.
        // All objects are from android.content.Context
        SharedPreferences settings = getSharedPreferences(PREFS_NAME, 0);
        SharedPreferences.Editor editor = settings.edit();
        editor.putBoolean("silentMode", mSilentMode);
        // Commit the edits!
        editor.apply();
        // use editor.commit() if you need the result value
    }
}
\end{lstlisting}
\end{breakbox}

\begin{breakbox}
\boxtitle{Interal Storage}

By default, files saved to the internal storage are private to your
application $\Rightarrow$ Other applications cannot access them.

To create and write a private file to the internal storage: 
\begin{itemize}
    \item Call
\textbf{openFileOutput()} with the name of the file and the operating
mode. This returns a FileOutputStream
\item Write to the file with write()
\item Close the stream with close()
\end{itemize}

\begin{lstlisting}
String FILENAME = "hello_file";
String string = "hello world!";
FileOutputStream fos = openFileOutput(FILENAME, Context.MODE_PRIVATE);
fos.write(string.getBytes());
fos.close();
\end{lstlisting}
\end{breakbox}

\begin{breakbox}
\boxtitle{External Storage}

\begin{itemize}
\tightlist
\item
  First you need to check the availability: getExternalStorageState()
\item
  Then you can access the data dir with File Streams
  getExternalFilesDir()
\end{itemize}

\begin{lstlisting}
File file = new File(getExternalFilesDir(null), "DemoFile.jpg");

try {
    // Very simple code to copy a picture from the
    // application's resource into the external
    // file. This code does no error checking, and
    // assumes the picture is small
    // (does not try to copy it in chunks).
    InputStream is = getResources().openRawResource(R.drawable.balloons);
    OutputStream os = new FileOutputStream(file);
    byte[] data = new byte[is.available()];
    is.read(data);
    os.write(data);
    is.close();
    os.close();
}
\end{lstlisting}
\end{breakbox}

\columnbreak
\begin{breakbox}
\boxtitle{Network Connection}

To send and receive data you may employ the class HttpURLConnection.

\begin{lstlisting}
URL url = new URL("http://www.android.com/");
HttpURLConnection urlConnection = (HttpURLConnection) url.openConnection();
try {
    InputStream in = new BufferedInputStream(urlConnection.getInputStream());
    readStream(in);
    finally {
        urlConnection.disconnect();
    }
}
\end{lstlisting}
\end{breakbox}

\begin{breakbox}
\boxtitle{Firebase}
\begin{lstlisting}
FirebaseDatabase database = FirebaseDatabase.getInstance();
DatabaseReference reference reference = database.getReference("url/to/node");

User user = new User(name, email);
reference.child("users").child(userId).setValue(user);

ValueEventListener postListener = new ValueEventListener() {
    @Override
    public void onDataChange(DataSnapshot dataSnapshot) {
        // Get Post object and use the values to update the UI
        Post post = dataSnapshot.getValue(Post.class);
        // ...
    }
    ...
\end{lstlisting}
\end{breakbox}
