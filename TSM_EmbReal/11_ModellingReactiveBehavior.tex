\hypertarget{modelling-reactive-behavior-notation---message-passing}{%
\section{Modelling Reactive Behavior Notation -
Message-Passing}\label{modelling-reactive-behavior-notation---message-passing}}

\begin{itemize}
\tightlist
\item
  Message passing is an asynchronous communication method
\item
  To communicate with the system environment via virtual interaction
\item
  Messages held in a buffer, e.g.~a message queue on the receiver side
  until they are handled
\end{itemize}

\begin{tcolorbox}[colback=red!5!white,colframe=red!75!black]
Ich habe den Nutzen dieses Kapitels nicht ganz verstanden und wir haben das Kapitel irgendwie einfach übersprugen bzw. in einer Minute angeschaut.
\end{tcolorbox}

\clearpage